\documentclass[12pt]{article}

\usepackage{amsmath}
\usepackage[margin = 1in]{geometry}
\usepackage{graphicx}
\usepackage{booktabs}
\usepackage{natbib}
\usepackage[colorlinks=true, citecolor=blue]{hyperref}

\title{Predicting Diabetes Based on Various Factors Using logistic Regression}
\author{Ashley Merritt\\
    Department of Statistics\\
    University of Connecticut
    }

\begin{document}
\maketitle

\begin{abstract}

\end{abstract}

\section{Introduction}
\label{sec:intro}
According to the American Diabetes Association, about 11.3 percent of Americans, or about 37.3 million people, have diabetes. Of those 37.3 million people, about 28.7 million were actually diagnosed with diabetes, while the remaining 8.6 million were left undiagnosed (\citet{CDC2022Diabetes}). Those left undiagnosed are at risk for even more serious illness if left untreated. Diabetes is a disease where your body does not create enough insulin, or use it properly, in order to get glucose into your cells and use it for energy (\cite{NIH2023Whatis}). Therefore, it is very important that your body has a normal glucose level. In this paper, I will be working to determine the pre-existing factors that can be used in order to predict if a person may have diabetes. Establishing the predictors is very important to me as many of my family members have a long history of suffering from this disease. It is important to better predict this disease to save people from unnecessary suffering. 

Previously, the prediction of diabetes has been studied using machine learning models. Specifically in Sisodia's paper, it focused on the "Prediction of Diabetes using Classification Algorithms", they worked through the topic by producing results using support vector machine, Naive Bayes classifier, and decision tree classifier (\cite{Sisodia2018Prediction}). In this paper, I will be continuing the investigation with a different data set using logistic regression and various other statistical tests.. 

In this paper the specific research question that I will be focusing on is: how can we use machine learning models in order to identify individuals that may have already or are at risk at developing diabetes? This will be researched using a data set that contains various metrics on one's health that will be described in the next section of the paper.

\section{Data}
\label{sec:data}
In order to study the proposed research question on the topic of diabetes, I searched through many databases in order to find a set that had all of the information I was looking for. After exploration, I found a data set entitled "Healthcare Diabetes Data set" on the website Kaggle. The data set was originally sourced from the National Institute of Diabetes and Digestive and Kidney Diseases. The data encompasses eight different predictors including pregnancies, glucose, blood pressure, skin thickness, insulin, BMI, diabetes pedigree function, and age.

The predictors listed above will be represented in my research as the following variables. 'Pregnancies' will provide the number of times an entry has been pregnant. 'Glucose' will provide the plasma glucose concentration over two hours using the results of an oral glucose tolerance test on a entry. 'BloodPressure' gives the diastolic blood pressure in mm Hg of an entry. 'SkinThickness' will provide the triceps skin fold thickness in mm. 'Insulin' will include a test on two hour serum insulin in mu U/ml. 'BMI' will provide a calculation of weight (kg) divided by height ($m^2$). 'DiabetesPedigreeFunction' will provide a genetic score of diabetes. 'Age' will simply be the entry's age at the time of the collection. 

Dataset link: \href{https://www.kaggle.com/datasets/nanditapore/healthcare-diabetes}{Healthcare Diabetes}

\begin{table}[ht]
    \centering
    \begin{tabular}{lrrr}
      \hline
    Variable & Mean & SD & Median \\ 
      \hline
    Id & 1384.50 & 799.20 & 1384.50 \\ 
      Pregnancies & 3.74 & 3.32 & 3.00 \\ 
      Glucose & 121.10 & 32.04 & 117.00 \\ 
      BloodPressure & 69.13 & 19.23 & 72.00 \\ 
      SkinThickness & 20.82 & 16.06 & 23.00 \\ 
      Insulin & 80.13 & 112.30 & 37.00 \\ 
      BMI & 32.14 & 8.08 & 32.20 \\ 
      DiabetesPedigreeFunction & 0.47 & 0.33 & 0.38 \\ 
      Age & 33.13 & 11.78 & 29.00 \\ 
      Outcome & 0.34 & 0.48 & 0.00 \\ 
       \hline
    \end{tabular}
    \caption{Descriptive Statistics} 
    \end{table}
    
\section{Methods}
\label{sec:meth}

\section{Results}
\label{sec:resu}

\section{Discussion}
\label{sec:disc}

\bibliography{refs}
\bibliographystyle{mcap}

\end{document}

